\documentclass[../main.tex]{subfiles} % required, if the Chapter be a seperate doc

\begin{document}
\\
\\
\\
    \section{Abstract}\label{sec:abstract}
        Diese interdisziplinäre Arbeit untersucht das Hookesche Gesetz anhand eines physikalischen Federexperiments.
        Ziel war es, die Federkonstanten zweier Federn in Einzel-, Seriell- und Parallelschaltung zu bestimmen und mit theoretischen Erwartungen zu vergleichen.
        Die Versuchsreihen umfassten systematische Längenmessungen bei definierter Gewichtszunahme, deren Daten statistisch ausgewertet wurden.
        \\
        \\
        Die Ergebnisse zeigten in allen Konfigurationen eine nahezu perfekte lineare Korrelation (R² ≈ 1.000) zwischen Dehnung und Kraft, was die Gültigkeit des Hookeschen Gesetzes bestätigt.
        Die experimentell ermittelten Federkonstanten entsprachen den theoretischen Vorhersagen: Einzel-Federn ~0.121 N/cm, Seriell ~0.060 N/cm, Parallel ~0.244 N/cm.
        \\
        \\
        Die Studie reflektiert zudem methodische Schwächen wie potenzielle Ablesefehler und begrenzte Messgenauigkeit.


\end{document}