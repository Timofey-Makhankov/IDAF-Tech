\documentclass[../main.tex]{subfiles} % required, if the Chapter be a seperate doc

\begin{document}

\chapter{Diskussion}\label{ch:diskussion}

    \section{Messwerte}\label{sec:messwerte}

       % \subsection{Formel des Federsystems}\label{subsec:formel-des-federsystems}

        Die lineare Beziehung $F = D \cdot \Delta l$ konnte in allen vier Setups bestätigt werden. Die R²-Werte lagen bei 1{,}000 – fast zu gut, um wahr zu sein. Die Federkonstanten stimmen mit den Erwartungen überein: Einzel-Federn ca. 0{,}121 N/cm, seriell halbiert (0{,}060 N/cm), parallel verdoppelt (0{,}244 N/cm).

        Kleine Abweichungen? Ja. Ursache? Möglicherweise Messungenauigkeiten oder Materialtoleranzen. Trotzdem: Hooke hatte recht – und unsere Daten zeigen’s.

    \section{Kritische Bewertung der Methoden}\label{sec:kritische-bewertung-der-methoden}

        %\subsection{Methodenkritik}

        Ablesen mit dem Doppelmeter: suboptimal. Kleinste Parallaxenfehler wurden durch Mittelwertbildung gemildert, aber nicht eliminiert. Die Gewichte? Ungeprüft – wir haben auf die Aufschrift vertraut. Der Aufbau, besonders bei Parallel, war nicht ganz stabil. Trotzdem: Die Statistik passt, die Modelle greifen.

        Fazit: solide, aber optimierbar. Für den Schulkontext? Absolut vertretbar.
\end{document}