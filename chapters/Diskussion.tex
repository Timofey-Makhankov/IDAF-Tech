\documentclass[../main.tex]{subfiles} % required, if the Chapter be a seperate doc

\begin{document}
\section{Diskussion}\label{sec:diskussion}

    \subsection{Messwerte}\label{subsec:messwerte}

        Die lineare Beziehung $F = D \cdot \Delta l$ konnte in allen vier Setups bestätigt werden.
        Die R²-Werte lagen bei 1{,}000 – fast zu gut, um wahr zu sein.
        Die Federkonstanten stimmen mit den Erwartungen überein: Einzel-Federn ca.0{,}121 N/cm, seriell halbiert (0{,}060 N/cm), parallel verdoppelt (0{,}244 N/cm).

        Kleine Abweichungen?
        Ja. Ursache?
        Möglicherweise Messungenauigkeiten oder Materialtoleranzen.
        Trotzdem: Hooke hatte recht – und unsere Daten belegen dies auch wunderbar.

    \subsection{Kritische Bewertung der Methoden}\label{subsec:kritische-bewertung-der-methoden}

        Ablesen mit dem Konstruktionslineal: nicht gerade optimal.
        Kleinste Ablesefehler wurden durch Mittelwertbildung gemildert, aber nicht eliminiert.
        Die Gewichte?
        Geprüft – wir haben auf die Aufschrift vertraut, aber auch selbst mit Wage und Newtonmeter nachgemessen.
        Der Aufbau, besonders bei den Parallelen Federn, war nicht ganz stabil, aber hat alles bis zum Schluss ausgehalten.
        Trotzdem: Die Statistik passt, die Modelle funktionieren.

        %\subsection{Methodenkritik}

        Ablesen mit dem Doppelmeter: suboptimal. Kleinste Parallaxenfehler wurden durch Mittelwertbildung gemildert, aber nicht eliminiert. Die Gewichte? Ungeprüft – wir haben auf die Aufschrift vertraut. Der Aufbau, besonders bei Parallel, war nicht ganz stabil. Trotzdem: Die Statistik passt, die Modelle greifen.

        Fazit: solide, aber optimierbar. Für den Schulkontext? Absolut vertretbar.
        Kurzgefasst: solide, aber optimierbar.
        Für den Kontext?
        Unserer Meinung nach Absolut vertretbar.


\end{document}